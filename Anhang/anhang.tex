\chapter*{Anhang}

{\huge Code HostStudi.java}

\begin{lstlisting}[language=Java,caption={Code HostStudi.java},label=lst:hoststudi,frame=single]
/*
* HostStudi.java
*
* by Nicolai Staege & Maik Maier
* DHBW Karlsruhe
* Germany
*
*/

package org.sunspotworld.demo;

import com.sun.spot.io.j2me.radiogram.*;
import com.sun.spot.peripheral.ota.OTACommandServer;
import com.sun.spot.util.IEEEAddress;
import com.sun.spot.util.Utils;
import java.awt.Color;
import javax.microedition.io.*;
import javax.swing.JFrame;
import javax.swing.JScrollPane;
import javax.swing.JTextArea;


/**
* Diese Klasse empfaengt die daten eines SPOTS
* um eine Bewegung anzuzeigen
* 
* @author: Nicolai Staege, Maik Maier
* 
*/
public class HostStudi extends JFrame {
	private static final int HOST_PORT = 67;
	private JTextArea status;
	private JFrame fr;
	private void setup() {
		fr = new JFrame("Einbruchsicherung");
		status = new JTextArea();
		JScrollPane sp = new JScrollPane(status);
		fr.add(sp);
		fr.setSize(360, 200);
		fr.validate();
		fr.setVisible(true);
	}
	
	private void run() throws Exception {
		RadiogramConnection rCon;
		Radiogram dg;
		try {
			// Lffnen eines serverseitigen Ports
			rCon = (RadiogramConnection) Connector.open("radiogram://:" + HOST_PORT);
			dg = (Radiogram)
			rCon.newDatagram(rCon.getMaximumLength());
		} catch (Exception e) {
		System.err.println("setUp caught " + e.getMessage());
		throw e;
	}
	
	status.append("Listening...\n");
	
	while (true) {
		try {
			// Empfange Datenpaket vom Spot
			rCon.receive(dg);
			long time = dg.readLong();      // lese Zeit
			System.out.println(time);
			status.append("Bewegung festgestellt!\n");
			
		} catch (Exception e) {
		System.err.println("Fehler " + e +  " beim Lesen des Pakets.");
		throw e;
	}
}
}

/**
* Hauptprogramm starten
*
* @param args any command line arguments
*/
public static void main(String[] args) throws Exception {
	OTACommandServer.start("HostStudi-GUI");
	HostStudi app = new HostStudi();
	app.setup();
	app.run();
}
}

\end{lstlisting}
\newpage

{\huge Code MovementDetection.java}
\begin{lstlisting}[language=Java,caption={Code MovementDetection.java},label=lst:hoststudi,frame=single]
/*
* MovementDetection.java
*
* by Nicolai Staege & Maik Maier
* DHBW Karlsruhe
* Germany
*
*/

package org.sunspotworld.demo;

import com.sun.spot.io.j2me.radiogram.*;
import com.sun.spot.resources.Resources;
import com.sun.spot.resources.transducers.
IAccelerometer3D;
import com.sun.spot.resources.transducers.ITriColorLED;
import com.sun.spot.resources.transducers.ILightSensor;
import com.sun.spot.util.Utils;
import java.io.IOException;
import javax.microedition.io.*;
import javax.microedition.midlet.MIDlet;
import javax.microedition.midlet.
MIDletStateChangeException;

/**
* Diese Klasse entdeckt die Bewegung des SPOTs an der Tuer und sendet
* daraufhin ein Signal an die Basestation, um zu signalisieren, dass
* die Tuer bewegt wurde.
* Die Desktop Applikation zeigt "Bewegung entdeckt" an.
*
* @author: Nicolai Staege, Maik Maier
* 
*/
public class MovementDetection extends MIDlet {

private static final int HOST_PORT = 67;

protected boolean isMoving() throws IOException {
	IAccelerometer3D accel = (IAccelerometer3D) Resources.lookup(IAccelerometer3D.class);
	return accel.getAccelZ() < 0 || accel.getAccelZ() > 0.1;
}

protected void startApp() throws MIDletStateChangeException {
	RadiogramConnection rCon = null;
	Datagram dg = null;

	String ourAddress = System.getProperty("IEEE_ADDRESS");

	ILightSensor lightSensor = (ILightSensor)
	Resources.lookup(ILightSensor.class);
	ITriColorLED led = (ITriColorLED)
	Resources.lookup(ITriColorLED.class, "LED7");


	new com.sun.spot.service.
	BootloaderListenerService().getInstance().start();

	try {
		// Oeffne Verbindung auf Port 67 - Port ist generell beliebig
		// Desktop Applikation muss allerdings den gleichen Port benutzen
		rCon = (RadiogramConnection) Connector.open("radiogram://broadcast:" + HOST_PORT);
		dg = rCon.newDatagram(50);  // Testsenden
	}
	catch (Exception e) {
		System.err.println("Exception " + e + " beim Verbindungsaufbau.");
		notifyDestroyed();
	}

while (true) {
	try {
	if (isMoving()) {
		// Uebermittle die aktuelle Zeit
		long now = System.currentTimeMillis();

		// LED leuchtet, um zu zeigen, dass sich der Sensor bewegt hat
		led.setRGB(255, 255, 255);
		led.setOn();
		Utils.sleep(50);
		led.setOff();

		// Zeit wird ins Package geschrieben, Package wird verschickt
		dg.reset();
		dg.writeLong(now);
		rCon.send(dg);
		}
	} 
	catch (Exception e) {
		System.err.println("Exception " + e + " waehrend des Senden eines Paketes.");
		}
	}
}

protected void pauseApp() {
// This will never be called by the Squawk VM
}

protected void destroyApp(boolean arg0) throws MIDletStateChangeException {
	// Only called if startApp throws any exception other than MIDletStateChangeException
	}
}
\end{lstlisting}

