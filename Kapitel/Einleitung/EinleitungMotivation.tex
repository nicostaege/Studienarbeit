\chapter{Einleitung und Intention}\label{c:Einleitung} %Label sind zum referenzieren

Der Computer ist mittlerweile zum festen Bestandteil im alltäglichen Leben geworden. Mit ihm können viele Aufgaben wie Recherchen, komplexe Rechnungen und Kommunikation vereinfacht und schnellstmöglich erledigt werden. Während vor einigen Jahren noch der Desktop-PC die beliebteste Wahl darstellte, geht der Trend mittlerweile in Richtung der mobilen Endgeräte wie z.B. Smartphone, Laptop oder auch Tablet. Menschen wollen sich nicht an einen Ort binden, an dem sie ihren Computer benutzen können und sehnen sich nach dem Wunsch, dass alle Alltagsgegenstände per Smartphone oder Tablet kontrollierbar werden.\\

Diese Vernetzung aller elektronischen Geräte in einem Haushalt wird als "Internet of Things" (kurz IoT) bezeichnet. Die grundsätzliche Idee besteht darin, dass alle elektronischen Geräte wie z.B. Kühlschrank, Backofen u.a. miteinander kommunizieren können und der Nutzer über sein mobiles Endgerät alle Daten der vernetzten Geräte einsehen und diese auch auf seinen Wunsch hin steuern kann. Nähere Informationen zu IoT folgen im nächsten Kapitel.\\

Zur beispielhaften Demonstration des Aufbaus eines solchen Netzes elektronischer Geräte beschäftigt sich diese Studienarbeit mit Oracle SunSpot, einem Sensornetzwerk bestehend aus 2 Sensoren und einer Basisstation. Im Folgenden wird die Inbetriebnahme und Programmierung dieser Sensoren vorgenommen und die darin enthaltene Technik erklärt. Ziel der Studienarbeit ist es, mit Hilfe von SunSpot eine Einbruchssicherung zu programmieren, indem bewegte Fenster oder Türen bei Abwesenheit des Besitzers der Wohnung erkannt werden, die Basisstation die Werte sammelt und sie an den Besitzer meldet.