\chapter{Einleitung und Intention}\label{c:Einleitung} %Label sind zum referenzieren

Der Computer ist mittlerweile zum festen Bestandteil im alltäglichen Leben geworden. Mit ihm können viele Aufgaben wie Recherchen, komplexe Rechnungen und Kommunikation vereinfacht und schnellstmöglich erledigt werden. Während vor einigen Jahren noch der Desktop-PC die beliebteste Wahl darstellte, geht der Trend mittlerweile in Richtung der mobilen Endgeräte wie z.B. Smartphone, Laptop oder auch Tablet. Menschen wollen sich nicht an einen Ort binden, an dem sie ihren Computer benutzen können und sehnen sich nach dem Wunsch, dass alle Alltagsgegenstände per Smartphone oder Tablet kontrollierbar werden.\\

Diese Vernetzung aller elektronischen Geräte in einem Haushalt wird als \textquotedblleft Internet of Things\textquotedblright (kurz IoT) bezeichnet. Die grundsätzliche Idee besteht darin, dass alle elektronischen Geräte wie z.B. Kühlschrank, Backofen u.a. miteinander kommunizieren können und der Nutzer über sein mobiles Endgerät alle Daten der vernetzten Geräte einsehen und diese auch auf seinen Wunsch hin steuern kann. Nähere Informationen zu IoT folgen im nächsten Kapitel.\\

Zur beispielhaften Demonstration des Aufbaus eines solchen Netzes elektronischer Geräte beschäftigt sich diese Studienarbeit mit Oracle SunSpot, einem Sensornetzwerk bestehend aus 2 Sensoren und einer Basisstation. Im Folgenden wird die Inbetriebnahme und Programmierung dieser Sensoren vorgenommen und die darin enthaltene Technik erklärt. Ziel der Studienarbeit ist es, mit Hilfe von SunSpot eine Einbruchssicherung zu programmieren, indem bewegte Fenster oder Türen bei Abwesenheit des Besitzers der Wohnung erkannt werden, die Basisstation die Werte sammelt und sie an den Besitzer meldet.

\section{Ziel dieser Arbeit}\label{s:Ziel} %Label sind zum referenzieren

Im Rahmen dieser Studienarbeit sollen theoretische Inhalte zum Thema Sensornetze behandelt sowie praktische Arbeiten unter Verwendung von Oracle SunSPOT erstellt werden. Im ersten Teil der Arbeit wird eine Wissensbasis geschaffen, welche Grundvoraussetzung zum Durchführen der praktischen Arbeiten ist. Diese beinhaltet theoretische Kenntnisse über das Internet Of Things, Wireless Sensor Networks, den IEEE-Standard 802.15.4, die in den SunSPOTS verbaute Hardware sowie die dazu benötigte Software.\\

Die durchgeführten praktischen Arbeiten werden im zweiten Teil dieser Arbeit aufgelistet und beschrieben. Es soll eine rudimentäre Einbruchserkennung realisiert werden, welche Bewegungen durch einen an Türen oder Fenstern angebrachten SunSPOT Sensor erkennt und die gesammelten Informationen an eine Host-Applikation sendet, welche die empfangenen Daten aufbereitet. Im Falle eines unerwünschten Eindringens in einen Raum soll eine entsprechende Meldung an den Wohnungsbesitzer gesendet werden.\\
Des weiteren soll ein Ausblick gegeben werden, inwieweit das im Rahmen der Studienarbeit erstellte Überwachungssystem erweiterbar ist. Zusätzlich soll die mögliche Integration weiterer Sensoren oder die Verwendung eines Raspberry Pi erläutert werden.\\

Gegen Ende der Arbeit sollen 'Smart Greenhouse' und 'Bot-So', beide Gewinner der 'IoT-Developer-Challenge', vorgestellt werden. Die beiden Projekte verfolgen ein vergleichbares Ziel wie das in dieser Studienarbeit vorgestellte Konzept und zeigen, wie das IoT zukünftig im täglichen Leben Einzug halten wird. \\

Abschließend werden die Ergebnisse der Arbeit kurz zusammengefasst und einem Fazit unterzogen. Dieses beinhaltet die Reflexion über die erreichten Ziele und persönliche Erfahrungen in Hinsicht auf die Arbeit mit Sensoren, ins besondere mit Oracle SunSPOT.