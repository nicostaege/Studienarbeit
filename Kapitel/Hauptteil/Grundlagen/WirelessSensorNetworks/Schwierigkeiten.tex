\subsection{Schwierigkeiten}\label{ss:Schwierigkeiten}

Beim Planen eines drahtlosen Sensornetzwerkes stellen sich vermehrt eine Schwierigkeiten heraus, die zu berücksichtigen sind. Viele dieser Schwierigkeiten sind auch untereinander abhängig und beeinflussen gleichzeitig die verwendete Elektronik, physikalische Aspekte, Rechenleistung oder auch Lebensdauer eines Sensors.\\

Die erste Schwierigkeit, die es bei einem Sensor zu betrachten gilt, ist die Versorgung des Sensors mit Energie. Dazu gibt es unterschiedliche Ansätze, wie z.B. die Versorgung des Sensors mit Batterien. Dies hat allerdings den Nachteil, dass deren Lebenszyklus beschränkt ist und Batterien früher oder später ausgetauscht werden müssen. Ein Akkumulator eignet sich hier schon besser, allerdings bleibt die Frage, wie der Akkumulator mit neuem Strom versorgt werden soll.\\

In Zusammenhang mit dieser Fragestellung ist die Betrachtung der möglichen Gewinnung oder Rückgewinnung von Energie durch den Sensor von Vorteil. Wie kann ein Sensor Energie gewinnen, ohne dass er damit direkt versorgt werden muss? Möglichkeiten wäre die Gewinnung von Solarenergie durch den Sensor, das Nutzen von Temperaturunterschieden in der Umwelt, die Rückgewinnung der Bewegungsenergie (z.B. durch Wind o.ä.) oder das Ausnutzen von Erschütterungen und Vibrationen. Diese Möglichkeiten sollten nur in Betracht gezogen werden, wenn der Sensor eine lange Lebensdauer mit sich bringen soll. Ist der Einsatz der Sensoren in absehbarer Zeit vorüber, lohnt es sich aus Produktions- und Kostengründen die begrenzte Lebenszeit des Sensors hinzunehmen, um ihn danach z.B. durch neue Sensoren auszutauschen.\\

Ein weiterer wichtiger Aspekt im Bereich Energie ist die Energieeffizienz. Dabei sollen alle Teile in einem Sensorknoten möglichst effizient nutzen, um die Energie sinnvoll und langsam aufzubrauchen. Nicht nur der Sensor selbst spielt dabei eine wichtige Rolle, sondern auch wie das Netzwerk um ihn herum aufgebaut und genutzt wird. Folgende Aspekte spielen bei der Energieeffizienz eine erhebliche Rolle:

\begin{itemize}
	\item Wahrnehmen der Daten durch die Sensoren
	\item Verarbeitung der Daten
	\item Sicherung der Daten
	\item Übertragung der Daten
	\item Empfang von Daten
\end{itemize} 

Damit der Energieverbrauch weiter eingeschränkt werden kann, sollten Sensoren nur aktiv sein, wenn sie wirklich benötigt werden. Ansonsten sollten sie sich in einen Sleep- bzw. Energiesparmodus begeben, um Energie zu sparen. Des Weiteren wäre zur ausschließlichen Wahrnehmung der Umgebung ein "Controller"- bzw. Sensormodus und zum Senden und Empfangen von Daten ein "Radio"- bzw. Übertragungsmodus von Vorteil.\\

Eine weitere Schwierigkeit, die sich stellt, ist der Einsatz bzw. die Verteilung der Sensoren in der Umwelt und ihre Selbstverwaltung. Die Knoten könnten entweder zufällig in der Umgebung platziert oder systematisch angeordnet werden. Hier entscheidet der jeweilige Anwendungszweck, wobei das systematische Platzieren der Sensoren meistens sinnvoller und effizienter ist.
Des Weiteren sollte unterschieden werden, ob aktive oder passive Sensoren eingesetzt werden sollen. Auch hier muss je nach Anwendungsfall unterschieden werden. Passive Sensoren eignen sich besser, wenn Daten nur erfasst und übermittelt werden sollen. Aktive Sensoren sollten eingesetzt werden, wenn auf die Erfassung der Daten eine eventuelle Aktion bzw. Reaktion mit der Umwelt erforderlich ist.\\ 

Sensoren sollten bestimmte Informationen über sich selbst und ihre Nachbarn wissen bzw. ermitteln können. Dazu gehören unter anderem ihre eigene Position, die Ortung der Nachbarknoten und ihre Identifikation, ihre eigene Knotenkonfiguration und ihre kürzeste Route zu einer Basisstation. Denn sobald ein Sensornetz einmal in Betrieb genommen wurde, muss es in der Lage sein, sich autonom betreiben und verwalten zu können. Dazu zählen die Anpassung an veränderte Umweltbedingung und das Kompensieren von Fehlern. Beim Ausfall eines Sensors soll das Sensornetz weiterhin aktiv und funktionsfähig bleiben.\\

Auch in Hinsicht auf die Sicherheit gibt es einige Aspekte zu betrachten. Manche Sensornetzwerke übertragen empfindliche und kritische Informationen, was sie zu einem beliebten Angriffsziel macht. Sie können sowohl von innen, von außen als auch direkt an den Knoten angegriffen werden. Es stellt sich als schwierig heraus, solche Netzwerke vor Angriffen zu schützen, da sie entfernt und selbstständig arbeiten, drahtlos kommunizieren und meistens keine speziellen Sicherheitsfeatures besitzen. Dies ist aus Energie-, Kostengründen und Gründen der Form und Größe der Sensoren meist nicht realisierbar. Übliche Sicherheitstechniken sind meist nicht durchführbar, da den Knoten üblicherweise die Rechen-, Kommunikations- und Speicherressourcen fehlen. Man braucht neu entwickelte Sicherheitsmechanismen für Sensornetze, die spezielle Lösungen für die Erkennung von Eindringlingen, Verschlüsselung, Schlüsselverwaltung und Verteilung und Registrierung von neuen Knoten besitzen, sodass die Ressourcen der Sensoren ausreichen und das Sicherheitskonzept realisierbar ist \cite{d:wolf}.