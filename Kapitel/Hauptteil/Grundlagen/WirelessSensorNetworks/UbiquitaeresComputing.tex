\subsection{Ubiquit\"ares Rechnen}\label{ss:UbiquitaeresRechnen}

1988 verwendete Mark Weiser erstmals den Begriff 'ubiquitous computing' (dt. ubiquitäres Rechnen), um seine Vision nach einem stets verfügbaren Rechensystem, welches dem Nutzer unsichtbar erscheinen soll, zum Ausdruck zu bringen. Der Computer soll sich so in den Alltag integrieren, dass die Menschen ihn gar nicht mehr bemerken. Nach seiner Vorstellung verbessere das ubiquitäre Rechnen die Erfahrungen, die man mit Computern macht, da die Rechner dem Nutzer nahtlos verfügbar gemacht werden, ohne dabei effektiv sichtbar zu sein.\\

Weiser zufolge sind die besten Technologien diejenigen, die scheinbar verschwinden, tatsächlich jedoch nur in den Hintergrund geraten und unsichtbar werden. Der Mensch soll nicht in der Welt des Computers leben, sondern der Rechner soll sich in die Welt des Menschen integrieren. In Lichtschaltern, Thermostaten, Stereoanlagen und Backöfen werden bereits heute kleine Rechner verbaut, die helfen sollen, den Alltag zu erleichtern und die Idee des 'Internet of Things' weiter zu verfolgen.\\

Da Ubiquitäres Rechnen zuverlässig und unsichtbar funktionieren soll, ist die Technologie der unsichtbaren Rechenmodule von großer Bedeutung. Voraussetzungen sind z.B. leistungsstarke Prozessoren, ausreichend Speicherplatz, drahtlose Kommunikation, Sensoren und Aktoren (die z.B. mit der Umwelt und dem Menschen interagieren). Der Mensch muss nicht für alle Anwendungsfälle von ubiquitärem Rechnen direkt eingebunden werden, da die Systeme auch autonom arbeiten können\cite{d:wolf}.\\