\section{IEEE 802.15.4}\label{ss:IEEE802154}

Das ‚Institute of Electrical and Electronics Engineers‘ (IEEE) definiert in seinem Standard 802.15.4 Protokolle für ein ‚Low Data Rate - Wireless Personal Area Network‘ (LR-WPAN). Die Art der Geräte, die für eine solche Art von Netz verwendet werden sollen, ist dabei nicht genauer spezifiziert. Sinnvoll ist der Einsatz vor allem bei Systemen wie Sensoren, Lichtquellen oder Schaltern, bei welchen eine niedrige Datenrate vollkommen ausreicht um miteinander zu kommunizieren. Grundsätzlich ist die Verwendung des Standards auch bei höherwertigen Geräten wie z.B. Handys oder anderen Multimediageräten möglich, allerdings ist die Sinnhaftigkeit eines solchen Einsatzes fragwürdig, da in Hinsicht auf die geringe Datenübertragungsrate viele Übertragungs- und Kommunikationsfunktionen solcher Geräte wohl kaum realisierbar wären. \\
Zur Kommunikation mehrerer Teilnehmer in einem nach 802.15.4-standardisierten Netzes werden die o.g. Topologien ‚Stern‘ und ‚Peer-to-Peer‘ unterstützt. Sollte die Spezifikation ‚ZigBee‘,  welche den 802.15.4 Standard erweitert, verwendet werden, können darüber hinaus auch vermaschte Netze realisiert werden. \\
Der Standard sendet laut Definition festgelegtem Frequenzspektrum auf den lizenzfreien Frequenzen 868-868,8 MHz (Europa), 902-928 MHz (Nordamerika, Australien) oder 2400 bis 2483,5 MHz (weltweit). Um die Frequenzen darüber hinaus zu spreizen, wird ein ‚Direct Sequence Spread Spectrum‘-Verfahren (DSSS) verwendet. //
Um den Zugriff auf das Medium untereinander zu koordinieren, wird CSMA/CA (‚Carrier Sense Multiple Access/Collision Avoidance‘) verwendet. Das bedeutet, dass, bevor ein Gerät mit einer Übertragung beginnt, überprüft wird, ob das Übertragungsmedium nicht bereits von einem anderen Endgerät genutzt wird. Ist das Medium nicht belegt, kann das Endgerät, welches das Medium überprüft hat, anfangen, seine Daten zu übertragen.\\
Der Standard erreicht Übertragungsraten zwischen 20 KB/s - 40 KB/s in den Frequenzbereichen von 868 MHz und 902-928 MHz, wohingegen im 2,4 GHz-Bereich Raten bis zu 250 KB/s realisiert werden können. Diese relativ geringen Datenraten zeigen bereits, dass der Standard nicht für eine Übertragung großer Daten konzipiert wurde. Es soll viel mehr eine energiesparende Übertragung geringer Datenmengen verwirklicht werden können.

\subsection{Komponenten}\label{ss:Komponenten}

\subsection{Unterstützte Topologien}\label{ss:UnterstutzeTopologien}

\subsection{Schichten}\label{ss:Schichten}

\subsection{Rahmenstruktur}\label{ss:Rahmenstruktur}

\subsection{Kommunikation}\label{ss:Kommunikation}

\subsection{Medienzugriff}\label{ss:Medienzugriff}

\subsection{Sicherheitsma"snahmen}\label{ss:Sicherheitsmassnahmen}