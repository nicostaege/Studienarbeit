\section{Adhoc-Netzwerke}\label{s:AdhocNetzwerke}

Adhoc-Netze sind in sich geschlossene Netzwerke, organisieren sich selbst und haben keine bestimmte Hierarchie. Sie bauen sich nur für die Dauer einer Datenübertragung auf, besitzen keine festgelegte Kommunikationsstruktur und verwalten und organiseren sich selbst. 

Adhoc-Netze sind leistungsfähige und zählen als so genanntes 'Self Organized Network' (SON), welche gute Lastverteilung betreiben und ohne zentrales Management auskommen. Die Endgeräte übernehmen in diesem Fall das Routing und speichern die Routingtabellen selbst ab. Geräte, die sich dem Netzwerk anschließen, werden dynamisch in das Netz eingefügt. Bei Netzwerken nach IEEE 802.11 (WLANs) und IEEE 802.15 (WPANs - hier im Speziellen IEEE 802.15.1 - Bluetooth) werden alle Geräte selbstständig erkannt und werden dem Netz hinzugefügt. Sie sind fortan Bestandteil des Gesamtnetzes. 

Bei Adhoc-Netzen mit vielen Geräten (das kann z.B. ein Sensornetz sein) wird zumeist eine Multihop-Verbindung bevorzugt. Das bedeutet, dass die Daten von einem Netzknoten, z.B. einem Sensor oder Rechner, zu dem nächsten Netzknoten weitergeleitet werden, bis es sein Ziel erreicht hat. Fällt ein Knoten aus, wird wenn möglich ein anderer Weg für die Übertragung genutzt, um Ausfälle zu vermeiden.

Ad-hoc-Netzwerke bestehen virtuell für einen begrenzten Zeitrahmen. Ad-hoc bedeutet etwa "für den Augenblick gemacht". Sie werden in WLANs, WPANs, in Sensornetzen (WPANs mit geringer Datenrate - siehe IEEE 802.15.4) und in Funknetzen von Rettungsdiensten, Polizei und Militär benutzt \cite{ws:lipinski}.

\subsection{MANETs}\label{ss:MANETs}
\subsection{Routingprotokolle}\label{ss:Routingprotokolle}