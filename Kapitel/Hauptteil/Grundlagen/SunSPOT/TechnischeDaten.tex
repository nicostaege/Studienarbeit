
\subsection{Gerätearten}\label{ss:Geraetearten}
Man unterscheidet bei den SunSPOT-Geräten grundsätzlich zwischen dem Basisstation SPOT und den Free Range SPOTs.\\

Der sogenannte 'Basestation SPOT' hat ein eSPOT Prozessor-Board ohne eigene Batterie. Er wird mit Hilfe des USB-Ports mit Strom versorgt. Die Station dient grundsätzlich als Schnittstelle zwischen Free Range SPOTs und dem PC (der Workstation) auf Basis von IEEE 802.15.4.\\

Die Free Range SPOTs enthalten ebenfalls ein Prozessor-Board (auch Mainboard genannt), hinzu kommt ein wiederaufladbarer Li-Ion Akkumulator. Des Weiteren enthalten die SPOTs in der Standardauslieferung das eDemo Sensor Board. Alle Boards, welche man dank modularem Aufbau hinzustecken kann, sind separat erhältlich und werden im Folgenden näher erläutert.

\subsection{Verfügbare Boards}\label{ss:VerfuegbareBoards}
Das sogenannte eSPOT Prozessor-Board besitzt in der aktuellsten Version eine 400 MHz 32-bit ARM CPU von Atmel, zusammen mit einem Flashspeicher von 8 Megabytes und einem Megabyte SRAM Hauptspeicher. Weiterhin ist es ausgestattet mit einem Radio Transceiver basierend auf IEEE 802.15.4 und einer USB 2.0 - Full Speed Schnittstelle. Der im SunSPOT integrierte Akkumulator hat eine Leistungsfähigkeit von 770mAh. Der maximale Energieverbrauch liegt zwischen 40-100 mA, abhängig von der Nutzung der integrierten LEDs, des Transceivers und anderer angeschlossener Geräte \cite{d:horan} \cite{d:spotmain}.  \\

Der Free Range SPOT wird dazu standardmäßig mit dem eDemo Sensor Board ausgeliefert. Es besitzt in der aktuellen Version einen 2G/4G/8G 3-Achsen-Beschleunigungssensor, einen Lichtsensor, 8 RGB 24bit LEDs, einen Infrarot-Sender \& Empfänger, ein kleiner Lautsprecher, 2 Knopfschalter, 4 analoge Eingänge, 4 I/O Pins, diverse weitere I$^2$C- und USART-Interfaces, einen EEPROM und 4 100mA Ausgangspins, mit denen es möglich ist, den SunSPOT-Sensor z.b. an weitere Lautsprecher oder andere Geräte anzuschließen \cite{d:horan} \cite{d:spotdemo}. \\

Weitere Boards, welche man nach Bedarf dazustecken kann, sind das eProto-Board, ein Board welches direkte Zugriffe auf das Prozessorboard ermöglicht und einen SD-Kartenslot besitzt, damit man die Daten dauerhaft speichern kann, das eSerial Board zum Verbinden via RS232 und das eFlash SD-Kartenleser Board \cite{d:horan}. \\