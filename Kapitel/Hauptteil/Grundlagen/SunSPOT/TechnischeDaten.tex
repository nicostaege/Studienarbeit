\subsection{Technische Daten}\label{ss:TechnischeDaten}

Das sogenannte eSPOT Prozessor-Board besitzt in der aktuellsten Version eine 400 MHz 32-bit ARM CPU von Atmel, zusammen mit einem Flashspeicher von 8 Megabytes und einem Megabyte SRAM Hauptspeicher. Weiterhin ist es ausgestattet mit einem Radio Transceiver basierend auf IEEE 802.15.4 und einer USB 2.0 - Full Speed Schnittstelle. Der im SunSPOT integrierte Akkumulator hat eine Leistungsfähigkeit von 770mAh. Der maximale Energieverbrauch liegt zwischen 40-100 mA, abhängig von der Nutzung der integrierten LEDs, des Transceivers und anderer angeschlossener Geräte. \cite{d:horan} \cite{d:spotmain} \\

Der SunSPOT-Sensor wird dazu standardmäßig mit dem eDemo Sensor Board ausgeliefert. Dieses Board besitzt in der aktuellen Version einen 2G/4G/8G 3-Achsen-Beschleunigungssensor, einen Lichtsensor, 8 RGB 24bit LEDs, einen Infrarot-Sender \& Empfänger, ein kleiner Lautsprecher, 2 Knopfschalter, 4 analoge Eingänge, 4 I/O Pins, diverse weitere I$^2$C- und USART-Interfaces, einen EEPROM und 4 100mA Ausgangspins, mit denen es möglich ist, den SunSPOT-Sensor z.b. an weitere Lautsprecher oder andere Geräte anzuschließen. \cite{d:horan} \cite{d:spotdemo} \\

Weitere Boards, welche man nach Bedarf dazustecken kann, sind das eProto-Board, ein Board welches direkte Zugriffe auf das Prozessorboard ermöglicht und einen SD-Kartenslot besitzt, damit man die Daten dauerhaft speichern kann, das eSerial Board zum Verbinden via RS232 und das eFlash SD-Kartenleser Board. \cite{d:horan} \\