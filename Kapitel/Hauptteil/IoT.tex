\chapter{Internet of Things}\label{c:IoT}

Als im Februar 1946 ENIAC, der erste elektronische sowie programmierbare Universalrechner vorgestellt wurde, wog dieser 27 Tonnen und füllte einen gesamten Raum. Anlagen dieser Größe wurden ausschließlich für wissenschaftliche Zwecke genutzt.
Mit der voranschreitenden Entwicklung werden Computer immer kleiner und leistungsfähiger. Es erschließen sich immer neue Anwendungen von Computersystemen, die hauptsächlich den Menschen in seinem Alltagsleben unterstützen sollen.
Das Haus wird durch ein komplexes Sicherheitssystem überwacht, die Tür benötigt nur den Fingerabdruck um sich automatisch zu öffnen, der Fernseher reagiert auf Spracheingaben und in der Zukunft erstellt der Kühlschrank autonom den Einkaufszettel.
Um all diese Daten gesammelt auswerten zu können sowie untereinander zu kommunizieren verbinden sich die Systeme mit dem Internet. Dieses ermöglicht einen Informationsfluss zwischen allen Teilnehmern. 
Das IoT ist entstanden.

\section{Geschichte}\label{s:gechichte}

Brereits 1991 schrieb Mark Weiser eine Vision, wie technische Geräte der Zukunft untereinander vernetzt sein könnten\cite{ws:weiser}.
Den Namen \ac{IoT} erhielt das ganze jedoch erst 1999.
\section{Ziele und Anwendungsbeispiele}\label{s:IoTZiele}

Aus Spielereien und purem Erfindergeist wurden in wenigen Jahren eine ganze Industrie, die sich heute nur mit Produkten des \ac{IoT} beschäftigt. Es entstanden bereits viele Projekte, denen man im Alltag begegnet, ohne sie Wahrzunehmen. 
Diese lassen sich in 3 Hauptkategorien unterteilen, die gleichzeitig die Ziele des \ac{IoT}s darstellen:

\begin{itemize}
	\item Automatisierung
	\item Informationsgewinnung über bessere Vernetzung
	\item Entertainment
\end{itemize}

In der folgenden Tabelle haben sind einige der Erfolgreichsten davon Zusammengestellt.

\begin{itemize} 
\item Umweltsensoren (Temperatur Feuchtigkeit Erschütterung Lautstärke Luftzusammensetzung) 
\item Lichtsteuerung
\item Haushaltshilfen
\item Bestandsaufnahme / Nachfuhrkontrolle
\item Überwachungsfunktionen
\item \glqq Smart Signs\grqq\ - Autobahn
\item Entertainment
\item Haussteuerung
\item Prozessüberwachung (Ventile, Flussraten usw.)
\item Diagnose / Lebensüberwachung usw.
\end{itemize}
\section{Sicherheitsaspekte}\label{s:Sicherheitsaspekte}

Über das Internet werden  immer mehr Informationen versendet. Ein kleiner Bestandteil hiervon sind auch die Informationen die verschiedene Geräte des \ac{IoT}s untereinander austauschen.
Diese automatisch abgewickelte \ac{M2M}-Kommunikation stellt ein hohes Sicherheitsrisiko für Unternehmen und Privatpersonen dar. Durch einen gezielten Angriff könnten personenbezogene oder sicherheitsrelevante Informationen an die Öffentlichkeit gelangen. Aufgrund dessen müssen sich die Verantwortlichen die Frage stellen, wie man mit dieser Herausforderung in der Zukunft umgeht. 

Sicherheitsexperten sind sich bereits heute einig, dass das \ac{IoT} ein enormes Risiko darstellt\cite{ws:iotsec}. Für Privatpersonen äußert sich das hauptsächlich in der Sicherheit ihrer Elektronischen Geräte. So musste der Automobilhersteller BMW kürzlich ein Softwareupdate für seine Automobile mit dem ConnectedDrive-System ausliefern, da es Hackern ohne Spuren zu hinterlassen gelungen war, die Türen mittels Smartphones zu öffnen \cite{ws:zeitbmw}.

