\chapter{Internet of Things}\label{c:IoT}

Als im Februar 1946 ENIAC, der erste elektronische sowie programmierbare Universalrechner vorgestellt wurde, wog dieser 27 Tonnen und füllte einen gesamten Raum. Anlagen dieser größe wurden ausschließlich für wissenschaftliche Zwecke genutzt.
Mit der voranschreitenden Entwicklung wurden Computer immer kleiner und leistungsfähiger. Es erschlossen sich immer neue Anwendungen von Computersystemen, die hauptsächlich den Menschen in seinem Alltagsleben unterstützen sollen.
Das Haus wird durch ein komplexes Sicherheitssystem überwacht, die Tür benötigt nur den Fingerabruck um automatisch aufzuschwingen, der Fernseher reagiert auf Spracheingaben und in der Zukunft erstellt der Kühlschrank den Einkaufszettel.
Um all die Daten zu sammeln, sowie untereinander zu kommunizieren verbinden sich die Systeme mit dem Internet. 
Das IoT ist entstanden.

\section{Geschichte}\label{s:gechichte}

Brereits 1991 schrieb Mark Weiser eine Vision, wie technische Geräte der Zukunft untereinander vernetzt sein könnten\cite{ws:weiser}.
Den Namen \ac{IoT} erhielt das ganze jedoch erst 1999.