\chapter{Internet of Things}\label{c:IoT}

Als im Februar 1946 ENIAC, der erste elektronische sowie programmierbare Universalrechner vorgestellt wurde, wog dieser 27 Tonnen und füllte einen gesamten Raum. Anlagen dieser größe wurden ausschließlich für wissenschaftliche Zwecke genutzt.
Mit der voranschreitenden Entwicklung wurden Computer immer kleiner und leistungsfähiger. Es erschlossen sich immer neue Anwendungen von Computersystemen, die hauptsächlich den Menschen in seinem Alltagsleben unterstützen sollen.
Das Haus wird durch ein komplexes Sicherheitssystem überwacht, die Tür benötigt nur den Fingerabruck um automatisch aufzuschwingen, der Fernseher reagiert auf Spracheingaben und in der Zukunft erstellt der Kühlschrank den Einkaufszettel.
Um all die Daten gesammelt auswerten zu können, sowie untereinander zu kommunizieren verbinden sich die Systeme mit dem Internet. Dieses ermöglicht einen Informationsfluss zwischen allen Teilnehmern. 
Das IoT ist entstanden.

\section{Geschichte}\label{s:gechichte}

Etwa um 1982 ärgerten sich drei Studenten der School of Computer Science, an der Carnegie Mellon University über den Getränkeautomaten ihres Instituts. An manchen Tagen liefen sie zu dem Automaten und erhielten entweder keine Getränke, oder zu warme, da diese erst kürzlich nachgefüllt wurden. Um diese Problematik zu lösen entwickelten die Studenten John Zsarnay neue Hardware die mit Software von David Nichols und Ivor Durham die Füllstände, sowie die Temperatur der Getränke überwachte\cite{ws:cmu}.\\
Über den damals an der Universität vorhandenen Vorgänger des Internets, das sogenannte Arpanet konnte direkt beim Automaten der aktuelle Status nachgefragt werden. Dieser antwortete zum Beispiel mit:\\

\begin{lstlisting}[frame=single] 
>                 EMPTY   EMPTY   1h 3m
>                 COLD    COLD    1h 4m
\end{lstlisting}

Hiermit informierte der Automat darüber, dass kalte Getränke in der mittleren sowie linken Schiene vorhanden seien, die Getränke der rechten Schiene jedoch noch warm seien. Die angegebene Zeit informierte darüber, wie lange die Getränke sich bereits im Automat befanden. Nach drei Stunden nahm der Automat an Getränke seien ausreichend gekühlt.\\


Bereits 1991 schrieb der Amerikanische Informatiker Mark Weiser eine Vision, wie technische Geräte der Zukunft untereinander vernetzt sein könnten\cite{ws:weiser}. 
Den Namen \ac{IoT} erhielt das ganze jedoch erst 1999.

\section{Ziele und Anwendungsbeispiele}\label{s:IoTZiele}

Aus Spielereien und purem Erfindergeist wurden in wenigen Jahren eine ganze Industrie, die sich heute nur mit Produkten des \ac{IoT} beschäftigt. Es entstanden bereits viele Projekte, denen man im Alltag begegnet, ohne sie Wahrzunehmen. 
Diese lassen sich in 3 Hauptkategorien unterteilen, die gleichzeitig die Ziele des \ac{IoT}s darstellen:

\begin{itemize}
	\item Automatisierung
	\item Informationsgewinnung über bessere Vernetzung
	\item Entertainment
\end{itemize}

In der folgenden Tabelle haben sind einige der Erfolgreichsten davon Zusammengestellt.

\begin{itemize} 
\item Umweltsensoren (Temperatur Feuchtigkeit Erschütterung Lautstärke Luftzusammensetzung) 
\item Lichtsteuerung
\item Haushaltshilfen
\item Bestandsaufnahme / Nachfuhrkontrolle
\item Überwachungsfunktionen
\item \glqq Smart Signs\grqq\ - Autobahn
\item Entertainment
\item Haussteuerung
\item Prozessüberwachung (Ventile, Flussraten usw.)
\item Diagnose / Lebensüberwachung usw.
\end{itemize}
\section{Sicherheitsaspekte}\label{s:Sicherheitsaspekte}

Hier wollen wir auf Sicherheitsaspekte des IoTs eingehen. Möglicherweise die Absicherung gegen Attacken von außen oder Ähnliches erläutern.