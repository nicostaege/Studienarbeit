\chapter{Internet of Things}\label{c:IoT}

Als im Februar 1946 ENIAC, der erste elektronische sowie programmierbare Universalrechner vorgestellt wurde, wog dieser 27 Tonnen und füllte einen gesamten Raum. Für private Anwendungen waren diese Rechnersysteme nicht geeignet. Mit der voranschreitenden Entwicklung werden Computer immer kleiner und leistungsfähiger. Es erschließen sich immer neue Anwendungen von Computersystemen, die hauptsächlich den Menschen in seinem Alltagsleben unterstützen sollen.

Rund um 1990, als das Internet kommerzialisiert und somit für jeden zugänglich wurde, begann eine rasante Entwicklung neuer Technologien. Bis heute hat es unsere Arbeitsweise sowie unser Privatleben verändert und dieser Trend schreitet ungehindert voran.\\

Mit dem Web 2.0 wurden Webseiten interaktiv sowie Videos und Bilder im Internet eine Selbstverständlichkeit. 
Soziale Netzwerke verbinden die Nutzer durch verschiedenste Arten der Kommunikation. Dieses sich immer weiter aufspannende Kommunikations- und Informationsnetz erreicht nun auch unsere kleinsten elektronischen Geräte.\\ 
Das Haus wird durch ein komplexes Sicherheitssystem überwacht, die Tür benötigt nur den Fingerabdruck um sich automatisch zu öffnen, der Fernseher reagiert auf Spracheingaben und in der Zukunft erstellt der Kühlschrank autonom den Einkaufszettel.

All diese Informationen werden über das Internet einer zentralen Sammelstelle oder dem menschlichen Akteur zugespielt, das \ac{IoT} ist entstanden.

\section{Geschichte}\label{s:gechichte}

Brereits 1991 schrieb Mark Weiser eine Vision, wie technische Geräte der Zukunft untereinander vernetzt sein könnten\cite{ws:weiser}.
Den Namen \ac{IoT} erhielt das ganze jedoch erst 1999.
\section{Ziele und Anwendungsbeispiele}\label{s:IoTZiele}

Aus Spielereien und purem Erfindergeist wurden in wenigen Jahren eine ganze Industrie, die sich heute nur mit Produkten des \ac{IoT} beschäftigt. Es entstanden bereits viele Projekte, denen man im Alltag begegnet, ohne sie Wahrzunehmen. 
Diese lassen sich in 3 Hauptkategorien unterteilen, die gleichzeitig die Ziele des \ac{IoT}s darstellen:

\begin{itemize}
	\item Automatisierung
	\item Informationsgewinnung über bessere Vernetzung
	\item Entertainment
\end{itemize}

In der folgenden Tabelle haben sind einige der Erfolgreichsten davon Zusammengestellt.

\begin{itemize} 
\item Umweltsensoren (Temperatur Feuchtigkeit Erschütterung Lautstärke Luftzusammensetzung) 
\item Lichtsteuerung
\item Haushaltshilfen
\item Bestandsaufnahme / Nachfuhrkontrolle
\item Überwachungsfunktionen
\item \glqq Smart Signs\grqq\ - Autobahn
\item Entertainment
\item Haussteuerung
\item Prozessüberwachung (Ventile, Flussraten usw.)
\item Diagnose / Lebensüberwachung usw.
\end{itemize}
\section{Hardware für Privatanwender}\label{s:DIY}

In den letzten Jahren sind viele Privatanwender aufgrund der sich bietenden Möglichkeiten dazu übergegangen, eigene Anwendungen mit kleinen \ac{IoT}-Systemen selbst zu entwickeln. Hierzu werden meist kleine Einplatinencomputer verwendet, da diese sehr günstig zu erhalten sind. 
Zwei der erfolgreichsten Einplatinencomputer sind die Modelle der Raspberry Pi und Arduino Familie.


\subsection{Arduino-Plattform}\label{ss:Arduino}

Arduino ist ein Unternehmen, dass all seine Produkte in enger Zusammenarbeit mit zugehörigen Community entworfen hat. Alle Entwicklungen basieren auf den Ideen und Konzepten, die die Gemeinschaft um das Unternehmen entwickelt hat. 

\begin{figure}[H] 
	\centering
	\includegraphics[scale=0.2]{Bilder/arduino}
	\caption{Das Arduino-Logo\cite{i:arduino}}
	\label{f:tree}
\end{figure}

Das Unternehmen entwickelt Mikrokontrollerplatinen, die ihre Umwelt durch Sensoren wahrnehmen und auf äußere Einflüsse reagieren können. Alle Produkte sind vorgefertigt oder als Selbstbau-Kit erhältlich. Da die Baupläne aller Platinen durch Arduino quelloffen im Internet erhältlich sind, können die Produkte auch von Grund auf nachentworfen werden.\\

Das erste Arduino-Board, dass 2005 vorgestellt wurde, hat bereits viele Revisionen erhalten. In der aktuellsten Version ist es mit einem 16\ac{MHz} 8-Bit Microcontroller ATmega328 der Firma Atmel bestückt. Es besitzt zusätzlich 32 Kilobyte Flash-Speicher von dem 0.5 Kilobyte durch den Bootloader belegt werden. Zur Ein- und Ausgabe, sowie zum anschließen von Sensoren besitzt die Platine 14 digitale Input- oder Output-Pins, sowie 6 analoge input Pins und einen \ac{USB}-Anschluss. Betrieben wird die Platine mit einem 7-12V Netzteil \cite{ws:arduinouno}.
Die Platine wird von dem italienischen Unternehmen SmartProjects produziert, ist jedoch aufgrund des quelloffenen Platinenplans auch selbst zusammensetzbar.\\

\begin{figure}[H] 
	\centering
	\includegraphics[scale=0.6]{Bilder/arduinouno}
	\caption{Die Front der dritten Arduino Uno Revision\cite{i:arduinouno}}
	\label{f:tree}
\end{figure}

Die Arduino-Plattform besitzt eine eigene \ac{IDE}, in der die Software für den Microcontroller entwickelt werden kann. Der Mikrocontroller wird mit einer stark vereinfachten Form der Programmiersprachen C und C++ angesprochen. Insbesondere technische Informationen wie Header-Files werden vor dem Programmierer verborgen. Für Anfänger bietet die \ac{IDE} mehrere Einstiegshilfen, die das Programmieren erleichtern.\\
Ein Beispiel für die Einstiegshilfen sind die strukturgebenden Funktionen setup(), sowie loop(). Während in der Setup-Methode alle Aktionen definiert werden müssen, die beim Programmstart durchgeführt werden sollen, wird der Code in der Loop-Methode unbegrenzt oft wiederholt.
Ein Beispielcode, der eine an Pin 13 angeschlossene LED blinken lässt, lautet wie folgt:\\

\begin{lstlisting}[language=c,caption={Simpler Arduino-Code, der eine LED blinken lässt},label=lst:blink,frame=single] 
int led = 13; //LED an Pin 13

void setup() {                
	pinMode(led, OUTPUT);     
}

void loop() {
	digitalWrite(led, HIGH);  
	delay(1000);               
	digitalWrite(led, LOW);   
	delay(1000);               
}
\end{lstlisting}

\vspace{1cm}

Seit der Entwicklung der Arduino-Boards erfreuen sich diese großer Beliebtheit. So wurden bis 2013 insgesamt 700.000 zusammengesetzte Platinen verkauft\cite{ws:sellnumb}.\\

Viele der Projekte, die mit Arduino-Platinen durchgeführt werden beinhalten Benutzerinteraktion. Auch für Kunstinstallationen werden die Einplatinencomputer gerne verwendet\cite{ws:elektor}.

\subsection{Raspberry Pi}\label{ss:Raspberry}


\section{Sicherheitsaspekte}\label{s:Sicherheitsaspekte}

Über das Internet werden  immer mehr Informationen versendet. Ein kleiner Bestandteil hiervon sind auch die Informationen die verschiedene Geräte des \ac{IoT}s untereinander austauschen.
Diese automatisch abgewickelte \ac{M2M}-Kommunikation stellt ein hohes Sicherheitsrisiko für Unternehmen und Privatpersonen dar. Durch einen gezielten Angriff könnten personenbezogene oder sicherheitsrelevante Informationen an die Öffentlichkeit gelangen. Aufgrund dessen müssen sich die Verantwortlichen die Frage stellen, wie man mit dieser Herausforderung in der Zukunft umgeht. 

Sicherheitsexperten sind sich bereits heute einig, dass das \ac{IoT} ein enormes Risiko darstellt\cite{ws:iotsec}. Für Privatpersonen äußert sich das hauptsächlich in der Sicherheit ihrer Elektronischen Geräte. So musste der Automobilhersteller BMW kürzlich ein Softwareupdate für seine Automobile mit dem ConnectedDrive-System ausliefern, da es Hackern ohne Spuren zu hinterlassen gelungen war, die Türen mittels Smartphones zu öffnen \cite{ws:zeitbmw}.



