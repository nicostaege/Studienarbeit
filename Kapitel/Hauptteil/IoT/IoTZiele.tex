\section{Ziele und Anwendungsbeispiele}\label{s:IoTZiele}

Aus Spielereien und purem Erfindergeist wurde in wenigen Jahren eine ganze Industrie, die sich heute nur mit Produkten des \ac{IoT} beschäftigt. Es entstanden bereits viele Projekte, denen man im Alltag begegnet, ohne sie wahrzunehmen. 
Diese lassen sich in 3 Hauptkategorien unterteilen, die gleichzeitig die Ziele des \ac{IoT}s darstellen:

\begin{itemize}
	\item Automatisierung
	\item Informationsgewinnung über bessere Vernetzung
	\item Entertainment
\end{itemize}

In vielen dieser Kategorien gibt es bereits umgesetzte Projekte. Manche davon begegnen uns bereits heute unbemerkt im Alltag.
In der folgenden Liste sind einige der Erfolgreichsten davon zusammengestellt.

\begin{itemize} 
\item Umweltsensoren (Temperatur Feuchtigkeit Erschütterung Lautstärke Luftzusammensetzung) 
\item Lichtsteuerung
\item Haushaltshilfen
\item Bestandsaufnahme / Nachfuhrkontrolle
\item Überwachungsfunktionen
\item \glqq Smart Signs\grqq\ - Autobahn
\item Entertainment
\item Haussteuerung
\item Prozessüberwachung (Ventile, Flussraten usw.)
\item Diagnose / Lebensüberwachung usw.
\end{itemize}