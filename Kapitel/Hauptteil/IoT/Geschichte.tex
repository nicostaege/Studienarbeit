\section{Geschichte}\label{s:gechichte}

Etwa um 1982 ärgerten sich drei Studenten der School of Computer Science, an der Carnegie Mellon University über den Getränkeautomaten ihres Instituts. An manchen Tagen liefen sie zu dem Automaten und erhielten entweder keine Getränke, oder zu warme, da diese erst kürzlich nachgefüllt wurden. Um diese Problematik zu lösen entwickelten die Studenten John Zsarnay neue Hardware die mit Software von David Nichols und Ivor Durham die Füllstände, sowie die Temperatur der Getränke überwachte\cite{ws:cmu}.\\
Über den damals an der Universität vorhandenen Vorgänger des Internets, das sogenannte Arpanet konnte direkt beim Automaten der aktuelle Status nachgefragt werden. Dieser antwortete zum Beispiel mit:\\

\begin{lstlisting}[frame=single] 
>                 EMPTY   EMPTY   1h 3m
>                 COLD    COLD    1h 4m
\end{lstlisting}

Hiermit informierte der Automat darüber, dass kalte Getränke in der mittleren sowie linken Schiene vorhanden seien, die Getränke der rechten Schiene jedoch noch warm seien. Die angegebene Zeit informierte darüber, wie lange die Getränke sich bereits im Automat befanden. Nach drei Stunden nahm der Automat an Getränke seien ausreichend gekühlt.\\


Bereits 1991 schrieb der Amerikanische Informatiker Mark Weiser eine Vision, wie technische Geräte der Zukunft untereinander vernetzt sein könnten\cite{ws:weiser}. 
Den Namen \ac{IoT} erhielt das ganze jedoch erst 1999.
