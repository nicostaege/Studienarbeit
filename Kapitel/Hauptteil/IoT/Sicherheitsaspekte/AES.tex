\subsection{Verschlüsselungsstandard AES}\label{ss:AES}

Der aktuell gültige Kryptographiestandard ist der im Jahr 2000 veröffentlichte \ac{AES}. Bei diesem Verfahren wird eine symmetrische Block-Cipher verwendet, die mit einem 128, 192 oder 256 Bits langen Schlüssel gleichlange Datenblöcke verschlüsselt\cite{ws:AES}.\\
Solange ein Verschlüsselungsverfahren wie \ac{AES} nicht durchbrochen wird, kann es nur durch sogenannte Brute-Force-Attacken angegriffen werden. Hierbei werden alle Möglichkeiten durchprobiert, bis man zufällig auf den richtigen Schlüssel trifft. Bei \ac{AES} mit 128 Bit Schlüssellänge sind hierfür bis zu $2^{128}\equiv 3,4*10^{38}$ Versuche nötig. Bei 256 Bit sind es bereits $2^{256}\equiv 1.2*10^{77}$ Versuche. Selbst mit Hilfe der aktuellsten Supercomputer würde dies Millionen Jahre in Anspruch nehmen.\\

\begin{table}[H] 
	\centering
	\begin{tabular}{|c|c|}\hline
		Schlüssellänge & Kombinationsmöglichkeiten\\ \hline \hline
		1-Bit & 2 \\ \hline
		2-Bit & 4 \\ \hline
		4-Bit & 16 \\ \hline
		8-Bit & 256 \\ \hline
		16-Bit & 65536 \\ \hline
		32-Bit & 4.2*10\textsuperscript{9} \\ \hline
		56-Bit (DES) & 7.2*10\textsuperscript{16} \\ \hline
		64-Bit  & 1.8*10\textsuperscript{19} \\ \hline
		128-Bit(AES) & 3.4*10\textsuperscript{38} \\ \hline
		192-Bit(AES) & 6.2*10\textsuperscript{57} \\ \hline
		256-Bit (AES) &  1.2*10\textsuperscript{77} \\ \hline
	\end{tabular}
	\caption{Kombinationsmöglichkeiten bei verschiedenen Schlüssellängen}
	\label{t:keylength}
\end{table}
\vspace{5 mm}

Jede in \ac{AES} erlaubte Schlüssellänge bietet aufgrund des komplexen Verschlüsselungsverfahrens und der daraus resultierenden Rechenzeit pro Angriff ausreichende Sicherheit. Da 128 Bit Schlüssel jedoch 40\% weniger Rechenleistung als 256 Bit Schlüssel benötigen, wird meist nur die minimale Schlüssellänge implementiert. 