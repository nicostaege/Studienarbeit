\section{Sicherheitsaspekte}\label{s:Sicherheitsaspekte}

Über das Internet werden  immer mehr Informationen versendet. Ein kleiner Bestandteil hiervon sind auch die Informationen die verschiedene Geräte des \ac{IoT}s untereinander austauschen.
Diese automatisch abgewickelte \ac{M2M}-Kommunikation stellt ein hohes Sicherheitsrisiko für Unternehmen und Privatpersonen dar. Durch einen gezielten Angriff könnten personenbezogene oder sicherheitsrelevante Informationen an die Öffentlichkeit gelangen. Aufgrund dessen müssen sich die Verantwortlichen die Frage stellen, wie man mit dieser Herausforderung in der Zukunft umgeht. 

Sicherheitsexperten sind sich bereits heute einig, dass das \ac{IoT} ein enormes Risiko darstellt\cite{ws:iotsec}. Für Privatpersonen äußert sich das hauptsächlich in der Sicherheit ihrer Elektronischen Geräte. So musste der Automobilhersteller BMW kürzlich ein Softwareupdate für seine Automobile mit dem ConnectedDrive-System ausliefern, da es Hackern ohne Spuren zu hinterlassen gelungen war, die Türen mittels Smartphones zu öffnen \cite{ws:zeitbmw}.\\

Zusätzlich zur Beschädigung oder Entwendung von Eigentum durch Hacker besteht ein Risiko private Informationen zu verlieren. Gerade Informationen wie E-Mail-Adressen oder Passwörter stehen in großem Interesse der Hacker. Solche könnten über ein privates \ac{IoT} entwendet werden und für weitere Cyber-Kriminelle Aktionen genutzt werden. \\

Auch Firmen müssen sich die Risiken bewusst machen, die sie durch die Benutzung von \ac{IoT}-Geräten eingehen. Oftmals nutzen Unternehmen \ac{IoT}-Systeme als Infrastrukturkomponenten über die sie einen Service betreiben. Solche Netze stellen einen Idealen Angriffspunkt für Wirtschaftsspionage oder Denial of Service - Attacken dar. Bricht ein Hacker in solche Systeme ein, wird es ihm schnell möglich sein einen Millionenschaden anzurichten.


Da die Entwicklung des \ac{IoT} derzeit noch an fahrt aufnimmt, muss jedem Nutzer bewusst sein, dass auch die Sicherheit noch nicht vollständig Entwickelt ist. Viele Aspekte werden sich in den nächsten Jahren weiterentwickeln, die Sicherheit wird jedoch zunehmend eine der wichtigsten Komponenten des \ac{IoT}s.\\


Jede am \ac{IoT} teilnehmende Komponente muss einen Zugang zur aufgebauten Netzwerkumgebung besitzen um ihre Metadaten an andere Teilnehmer zu kommunizieren. Über diese zusammengesammelten Informationen trifft das System eine Entscheidung zu handeln. Hierbei entsteht ein Sicherheitsrisiko.\\
Mit netzwerkfähigen Geräten ist es möglich, sich in das aufgebaute Netzwerk einzuklinken, und sich ebenfalls als Komponente auszugeben. Dies ermöglicht einem Angreifer Daten abzufangen oder zu seinen Zwecken zu manipulieren. Dies kann das verhalten des Gesamten Systems verändern.\\
Ebenfalls ist es möglich, dass ein Angreifer ein Teilnehmer des \ac{IoT}s übernimmt und somit Zugriff auf das Netzwerk erhält.