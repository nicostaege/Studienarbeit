\section{Sicherheitsaspekte}\label{s:Sicherheitsaspekte}

Über das Internet werden  immer mehr Informationen versendet. Ein kleiner Bestandteil hiervon sind auch die Informationen die verschiedene Geräte des \ac{IoT}s untereinander austauschen.
Diese automatisch abgewickelte \ac{M2M}-Kommunikation stellt ein hohes Sicherheitsrisiko für Unternehmen und Privatpersonen dar. Durch einen gezielten Angriff könnten personenbezogene oder sicherheitsrelevante Informationen an die Öffentlichkeit gelangen. Aufgrund dessen müssen sich die Verantwortlichen die Frage stellen, wie man mit dieser Herausforderung in der Zukunft umgeht. 

Sicherheitsexperten sind sich bereits heute einig, dass das \ac{IoT} ein enormes Risiko darstellt\cite{ws:iotsec}. Für Privatpersonen äußert sich das hauptsächlich in der Sicherheit ihrer Elektronischen Geräte. So musste der Automobilhersteller BMW kürzlich ein Softwareupdate für seine Automobile mit dem ConnectedDrive-System ausliefern, da es Hackern ohne Spuren zu hinterlassen gelungen war, die Türen mittels Smartphones zu öffnen \cite{ws:zeitbmw}.\\

Zusätzlich zur Beschädigung oder Entwendung von Eigentum durch Hacker besteht ein Risiko private Informationen zu verlieren. Gerade Informationen wie E-Mail-Adressen oder Passwörter stehen in großem Interesse der Hacker. Solche könnten über ein privates \ac{IoT} entwendet werden und für weitere Cyber-Kriminelle Aktionen genutzt werden. \\

Auch Firmen müssen sich die Risiken bewusst machen, die sie durch die Benutzung von \ac{IoT}-Geräten eingehen. Oftmals nutzen Unternehmen \ac{IoT}-Systeme als Infrastrukturkomponenten über die sie einen Service betreiben. Solche Netze stellen einen Idealen Angriffspunkt für Wirtschaftsspionage oder Denial of Service - Attacken dar


DO NOT USE THIS: IS COPIED:




Die Sicherheitsexperten sind sich heute bereits einig, dass die Sicherheitsbedrohungen des Internet der Dinge enorm hoch sind und diese möglicherweise sogar die Verbreitung dieser Systeme erheblich verzögern kann. Da die IoT-Systeme wichtige Infrastrukturkomponenten darstellen, sind diese natürlich ein hervorragendes Ziel für eine Wirtschaftsspionage, Denial of Service- und andere Angriffen. Ein weiterer wichtiger Aspekt ist die bereits von den Datenschützern geäußerte Sorge, dass durch IoT noch mehr persönliche Daten in den Netzen verfügbar sein werden, die möglicherweise Ziele für Cyber-Kriminelle darstellen.

Man muss bei Bewertung der Sicherheitsanforderungen für das Internet der Dinge jedoch im Auge behalten, dass sich diese Technologie erst im Anfangsstadium befindet und viele Aspekte heute noch nicht absehbar sind. Das IoT wird sich auch nicht über Nach entwickeln, sondern allmählich an Bedeutung gewinnen. Einige Dinge sind bereits mit dem Internet verbunden, aber wir werden zukünftig einen Boom an autonomen Maschinen erleben, die eigenständig ihre Informationen untereinander austauschen.

Die erwartete explosionsartige Zunahme der IoT-Komponenten und -Objekten, welche mit anderen Maschinen/Objekten bzw. Menschen in Echtzeit (oder annähender Echtzeit) ihre Daten austauschen, muss die Sicherheit in einer M2M-Welt zu einem festen Bestandteil der Lösung werden.

Zu den wichtigsten Sicherheitsüberlegungen bei IoT gehört, dass jedes Objekt, sei es ein LKW, ein Automaten oder eine Medizinflasche, zu ein Teil der Netzwerkumgebung wird. Das IOT sorgt für die virtuelle Präsenz eines physischen Objekts. Diese virtuelle Präsenz interagiert mit anderen Geräten und tauscht Kontextinformationen aus. Auf Basis dieser Informationen treffen die Geräte ihre logischen Entscheidungen. Da diese Objekte zukünftig fester Bestandteil einer vernetzten Umgebung sind, müssen wir erkennen, dass diese Geräte ihre physische Sicherheit verlieren. Viele diese Geräte werden jedoch in unwirtlichen Umgebungen installiert und können dadurch von nicht autorisierten Person leicht erreicht werden. Angreifer können die Daten möglicherweise abfangen, mitlesen oder verändern. Mit handelsüblichen IoT-Geräten lassen sich die Daten somit für jeden beliebigen Zweck manipulieren. Dies ist bei der Risikobetrachtung dieser neuen Komponenten einzukalkulieren.

Vor Jahren wurden die klassischen PC-Drucker gegen netzwerkfähige Geräte ausgetauscht. Dadurch wurde aus einem klassischen Arbeitsgerät ein Sicherheitsrisiko. Jeder Netzwerkdrucker verfügt über einen integrierten (Web) Server. Greift man über einen Browser auf das Gerät zu, erhält man die komplette Steuerung über das Gerät. Standardmäßig verfügen die meisten Netzwerkdrucker über ein "Blank"-Passwort. Die Passworte können geändert werden, wenn der Administrator seine Hausaufgaben korrekt erledigt. Ein weiteres Problem mit diesen Geräten ist das Patch/Upgrade-Management. Die integrierten Server sind anfällig für Angriffe. Ein Patch/Upgrade eines in einem netzwerkfähigen Druckers ist keine einfache Sache. In der Regel muss hierbei ein Firmware-Upgrade vorgenommen werden. Damit ist der Besitzer/Administrator auf den jeweiligen Hersteller des netzwerkfähigen Druckers angewiesen. Stellt dieser keine neue Firmware zur Verfügung, dann bleiben die Sicherheitsprobleme im Drucker bestehen. Die integrierten Server und die Probleme der Patches/Upgrades sind zwei Problembereiche, die auch bei IoT-Komponenten auf uns zukommen.

Es geht nicht um Panikmache, denn Sicherheitsvorfälle von IoT-Implementierungen sind bereits bekannt geworden. Die meisten Beispiele stammen jedoch aus Labor -oder Testumgebungen. Da diverse Bedrohungen bereits Realität sind, wundert es doch, dass diese in der Öffentlichkeit kaum wahrgenommen werden. Erst kürzlich haben Forscher zwei Autos gehackt und drahtlos die Bremsen deaktiviert und die Lichter ausgeschaltet - alles außerhalb der Kontrolle des Fahrers. In einem anderen Fall wurde das GPS-Signal manipuliert und die Frachtschiffe auf einen falschen Kurs gelockt. Auch die Home Control Hubs sind sehr verletzlich, so dass die Angreifer die Heizung, die Beleuchtung, die Stromzufuhr und die Türschlösser manipuliert werden können. Auch lassen sich industriellen Steuerungssysteme ist das drahtlose Netzwerk und der Sensoren leicht hacken.

Gehackt Fernseher, Videokameras und Baby-Phones gehören heute quasi zum Alltag und hegen natürlich die Bedenken gegenüber diesen Geräten, hinsichtlich der Privatsphäre und der Datensicherheit. Früher manipulierte man die Stromzähler im Haus um "kostengünstig" an elektrische Energie zu kommen. Heute muss man nur noch über das Internet in diese Geräte einbrechen und schon ist die nächste Stromrechnung "optimiert.

In einem kürzlich erschienenen Artikel sprach der Autor von einer "gehackt Glühbirne".  So absurd diese Vorstellung klingt, so realitätsnah sind jedoch deren Wirkungen. Ein Wurm, der eine große Anzahl von diesen im Internet verfügbaren IoT-Geräten infiziert, kann über ein Botnetz jede beliebige Information einsammeln und für diese für seine Zwecke missbrauchen. Ein weiteres Angriffsszenario zielt auf die verfügbare Bandbreite ab. Mit einem gezielten DDoS (Distributed Denial-of-Service) Angriff lassen sich alle IT-Komponenten vom Netz nehmen. Oder man stelle sich vor, alle kompromittierten IoT-Geräte in einem Unternehmen greifen gezielt mit hohen Anfrageraten auf die Steuerungsrechner zu und bringen diesen zum Erliegen bzw. zum Absturz.

Das Internet der Dinge schafft somit relativ komplexe sicherheitspolitische Herausforderungen für die Unternehmen. Als autonome Maschinen sind diese in der Lage, mit anderen Maschinen zu interagieren und selbständig Entscheidungen zu treffen. Diese Entscheidungen haben direkte Auswirkungen auf die reale Welt. Wir kennen ähnliche Probleme mit automatischen Trading-Systemen in Banken. Ein kleines Softwareproblem kann bereits zum Absturz ganzer Börsensystem führen und kann zu Folgekosten im Millionenbereich führen.

Alle IT Systeme lassen sich fehlertolerant auslegen bzw. aufbauen. Da die genutzte Software von Menschen die fehlbar sind, programmiert wurden, können auch auf einem vermeintlich "sicheren" System immer wieder Fehler auftreten.

Sicherheitsbedrohungen von IoT-Systemen haben direkte Auswirkungen auf eine Menge von Menschen. Wenn die Sicherheit eines der aktuellen IT-System ausfällt können werden eventuell ein paar hundert Kreditkartendaten gestohlen oder ein Politiker bloßgestellt. Alles kein großes Problem. Stellen wir uns stattdessen vor, ein zentrales Leistungssystem wird gehackt und die Einbrecher schaltete das Licht eines Teils oder einer ganzen Stadt während der Morgenstunden aus. Tausende von Menschen stecken auf einmal in den U-Bahnen unter der Erde in völliger Dunkelheit fest und die Produktionsbetriebe dieser Stadt können keine Waren mehr herstellen. Der Unterschied zwischen den beiden Angriffsszenarien massiv. Mit Hilfe von IoT interagiert die virtuelle Welt mit der physischen Welt und dies erfordert ein höheres Sicherheitsniveau.

In der derzeitigen IoT-Entwicklungsphase werden noch traditionelle Sicherheitslösungen im IoT-Markt vorausgesetzt. Das Netzwerk stellt in diesem Fall noch alle Sicherheitsfunktionen bereit. Erst in der nächsten Entwicklungsphase wird die Netzwerksicherheit durch zusätzliche IoT-Sicherheitsfunktionen (welche auf Chip-Ebene, auf einer SIM-Karte oder auf M2M-Modulen integriert sind) ergänzt. Ein Zukunftsszenario sieht vor, dass die M2M-Anwendungen eines Tages sämtliche Sicherheitsrisiken abdecken und eine vollkommen eigenständige Sicherheitslösung bereitstellen.

Das Internet der Dinge formuliert wesentliche Fragen zur Sicherheit neu. Den Verlust der Privatsphäre, die künftige Vermischungen von persönlichen und betrieblichen Daten und das Monitoring. Der Verlust der Privatsphäre hängt vom jeweiligen Aufenthaltsort (im Unternehmen, Unterwegs oder Zuhause) der betreffenden Person und dessen Handlungen (beispielsweise welches Produkt die Person gerade erwirbt) ab. Die meisten Nutzer von Handys halten diese Geräte rund um die Uhr betriebsbereit und sind permanent im Mobilfunknetz eingebucht. Anhand der Mobilfunkmasten lassen sich unsere Bewegungen bereits heute permanent verfolgen. Anhand der aktuellen Verbrauchsdaten eines intelligenten Stromzählers lässt sich darauf schließen, ob ein Nutzer sich in seinem Haus oder Unterwegs befindet. Der Stromverbrauch lässt auch Rückschlüsse zu, ob wir Nachtschwärmer oder Frühaufsteher sind.

Die Trends des ITK-Markts wie beispielsweise Cloud Computing, Mobilität und Big Data, wirkt sich auf die Sicherheitsanforderungen der Unternehmen und den Risikobetrachtungen direkt aus und beeinflussen auch die Sicherheitsanforderungen der M2M-Architekturen. Bei der Vermischungen von persönlichen und betrieblichen Daten stehen wir vor der gleiche Herausforderung, wie wir sie vom Einsatz der mobilen Technologien am Arbeitsplatz und dem Bring Your Own Device (BYOD) Trend her kennen. Smartphones gehören heute zum Arbeitsalltag. In manchen Fällen werden die Geräte vom Unternehmen, in anderen Fällen vom jeweiligen Nutzer beschafft. Natürlich gibt es inzwischen auch für die BYOD-Probleme technische Lösungen, wie beispielsweise die Datenverschlüsselung und die Möglichkeit zum Remote Löschen von Informationen. Letztere Lösung wirft jedoch einige rechtliche Probleme auf: Darf das Unternehmen legal die Daten eines Anwenders auf dessen privaten Gerät löschen?

Die enorme Anzahl von IoT-Geräten und -Gegenständen, die zukünftig durch M2M in die Kommunikationsnetze angeschlossen werden, integrieren die Unternehmensabläufe noch tiefer in die Unternehmenskommunikation. Big Data wird nicht nur ein Schlagwort ohne Hülle sein, sondern durch das Internet der Dinge mittelfristig in jedem Unternehmen an Relevanz gewinnen. Aus diesem Grund ist es notwendig, dass sich die Unternehmen auf Basis einer wohldefinierten Risikoanalyse die Themen Datensicherheit, Vertrauen und Privatsphäre untersuchen und auf Basis dieser Ergebnisse eine entsprechende Sicherheitspolitik ableiten.

Die von den Marktanalysten für das Jahr 2020 vorausgesagten 50 Milliarden IoT-Komponenten legen die Grundlage für eine umfassend "vernetzte Gesellschaft". In den verbleibenden sieben Jahren, müssen daher die  IT-Branche sowie die zuständigen Standardisierungsgremien für tragfähige und nachhaltige Rahmenbedingungen schaffen, damit sich die Bedürfnisse der "digitalen Unternehmen" und der "digitalen Bürger" in der nächste Generation von M2M/IoT-Lösungen wiederfinden.

Sicherheitsnormen für das Internet der Dinge
Das Internet der Dinge beruht auf dem Konzept der Identitäten. Diese schließen sowohl reale Nutzer als auch "Dinge" (Objekte) ein. Daher erscheint es zwingend notwendig, also, dass sich die Hersteller von IoT/M2M-Komponenten auf verbindliche Identity-Standards einigen. Dies beinhaltet die allgemeingültige Definition von Elementen und deren Identitäten, ein standardisiertes Modell zur Objekt-Identifikation und Authentifizierung. Erst die Verabschiedung von Standard-Kommunikationsprotokollen für IoT ebnet den Weg in den Mainstream. Beispiele hierfür sind das Message Queuing Telemetrie Traffic (MQTT) Protokoll, dem M2M Äquivalent zu HTTP. Ebenso müssen Mechanismen gefunden werden, die eine Interoperabilität zwischen verschiedenen Objekten und Services garantiert

Es müssen die Fehler der Datenkommunikation vermieden werden. Die Sicherheit darf nicht mehr nachträglich an die eigentlichen Kommunikationsfunktionen angeflanscht  werden.

Sicherlich werden die Sicherheitsfunktionen nicht von Anfang an in die IoT-Objekte integriert werden. Dies hat oft den Grund, dass meist ein Mangel an lokalen Ressourcen oder Kapazitäten besteht. Die Sicherheit wird daher zuerst in den zuständigen Web-Diensten implementiert werden, welche direkt vor dem Objekt arbeiten und deren Funktionalität bereitstellen. Die jeweiligen Objekte werden sich auf die Themen Integrität der Nachrichten und der Absicherung der Kommunikationsflüsse konzentrieren. Mit den Entwicklungsfortschritten  . wird die Sicherheit näher an das Objekt heranrücken und eines Tages direkt auf der Chip-Ebene eingebettet werden. 