\subsection{Umsetzung}\label{s:Umsetzung}

Im Rahmen der Studienarbeit wurden mit der NetBeans IDE und der Programmiersprache Java zwei Applikationen erstellt, welche einen ersten Ansatz für eine Einbruchsicherung realisieren. Sie sind beliebig erweiterbar, so dass diese als Grundlage für eine komplexere Raumüberwachung dienen können. Die Gesamtimplementation wurde in zwei Teilapplikationen aufgeteilt. Die erste Applikation ist die 'Desktop-Applikation‘, welche auf dem Rechner läuft und die Daten vom SunSPOT-Sensor empfängt. Auf dem SunSPOT läuft der zweite Teil der Implementation, welcher Bewegungen des Sensors feststellt und diese an eine SunSPOT-Basisstation weiterleitet. In den folgenden Abschnitten wird die genauere Funktionsweise der zwei Teilapplikationen behandelt.

\subsubsection{Desktop-Applikation}\label{sss:Desktop-Applikation}

Die Desktop-Applikation der praktischen Arbeit umfasst knapp 100 Zeilen und dient zum Aufbau einer Radiogram-Verbindung zwischen SunSPOT und Basisstation sowie zum Empfang der vom SPOT gesendeten Daten. \newpage

\begin{lstlisting}[language=Java,caption={Ausschnitt aus der setup()-Methode},label=lst:setup,frame=single] 
private void setup() {
	fr = new JFrame("Einbruchsicherung");
	status = new JTextArea();
	JScrollPane sp = new JScrollPane(status);
	fr.add(sp);
	fr.setSize(360, 200);
	fr.validate();
	fr.setVisible(true);
}           
\end{lstlisting}

Im ersten Schritt wird die Methode ‚setup()‘ definiert, welche zur besseren Visualisierung der Informationen einen JFrame erstellt, in welchen nachher eingetragen werden kann, dass der Sensor eine Bewegung erfahren hat. Sie wird zum Start des Programms aufgerufen, damit das Fenster auch entsprechend erstellt wird. \\

Des Weiteren ist eine Methode ‚run()‘ vorhanden, welche direkt nach der ‚setup()‘-Prozedur aufgerufen wird. Diese Funktion kümmert sich sowohl um das Öffnen eines serverseitigen Ports als auch um den Empfang von Datenpaketen, welche an diesen Port versendet werden. \\
Um Daten vom Sensor empfangen zu können, muss in der Desktop-Applikation auf einem selbst festgelegten Port nach Verbindungen 'gelauscht‘ werden. 
\\
\begin{lstlisting}[language=Java,caption={Öffnen des serverseitigen Ports},label=lst:portserver,frame=single] 
try {
	// Oeffnen einer Radiogram-Verbindung auf der Desktop-Seite
	// um Daten von Sensoren empfangen zu koennen
	rCon = (RadiogramConnection) Connector.open("radiogram://:" + HOST_PORT);
	dg = (Radiogram)
	rCon.newDatagram(rCon.getMaximumLength());
	} 
	
catch (Exception e) {
	System.err.println("setUp caught " + e.getMessage());
	throw e;
}
\end{lstlisting}

Der oben gezeigte Codeausschnitt eröffnet einen serverseitigen Port mit Hilfe der vorher festgelegten HOST\_PORT Variable und lauscht nun fortan auf diesem für Pakete, die vom SunSPOT an die Basisstation versendet werden.\\

\begin{figure}[H] 
	\centering
	\includegraphics[scale=0.7]{Bilder/server}
	\caption{Logdaten der Basisstation bei Servererstellung}
	\label{f:server}
\end{figure}

Der zweite Teil der ‚run()‘-Methode besteht aus einer while-Schleife, welches auf ankommende Pakete wartet und diese entgegennimmt. Nur wenn der SPOT eine Bewegung erfährt, sendet er ein Paket an die Basisstation. So kann die Desktop-Applikation sicher sein, dass immer wenn ein Paket gesendet wird eine Bewegung stattfand.\\

\begin{lstlisting}[language=Java,caption={Auswerten des empfangenen Paketes},label=lst:rcvpackage,frame=single] 
while (true) {
	try {
		// Empfangen und Lesen des Pakets
		rCon.receive(dg);
		long time = dg.readLong();
		System.out.println(time);
		status.append("Bewegung festgestellt!\n");
		} 
	catch (Exception e) {
		System.err.println("Exception " + e +  " beim Lesen der Daten.");
		throw e;
		}
	}
\end{lstlisting}

Innerhalb des übermittelten Paketes befindet sich die aktuelle Uhrzeit, um die Bewegung nachher genau zeitlich einordnen zu können. Sobald die Desktop-Applikation ein Paket empfängt, gibt sie auf dem Bildschirm innerhalb des erstellten Frames den Text 'Bewegung festgestellt!' aus. So wird dem Nutzer unmissverständlich klar gemacht, dass der Sensor in Bewegung war.

\begin{figure}[H] 
	\centering
	\includegraphics[scale=1.0]{Bilder/bewegung}
	\caption{JFrame mit textueller Ausgabe}
	\label{f:bewegung}
\end{figure}

Innerhalb der Konsole werden neben Debugausgaben zur Basisstation und die Liste eröffneter Ports der Zeitstempel, welcher mit der Bewegung vom SunSPOT übermittelt wird, ausgegeben.

\begin{figure}[H] 
	\centering
	\includegraphics[scale=0.5]{Bilder/timestamp}
	\caption{Konsolenausgabe mit Timestamp}
	\label{f:timestamp}
\end{figure}

\subsubsection{SPOT-Applikation}\label{sss:SPOT-Applikation}

Der Desktop-Applikation gegenüber steht die Applikation, welche auf dem SunSPOT betrieben wird. Diese erkennt Bewegungen des SPOTs und sendet diese über eine Radiogram-Verbindung an die auf dem selben Port lauschende Basisstation.

\begin{lstlisting}[language=Java,caption={Aufbau der Verbindung auf Port 67},label=lst:spotport,frame=single] 
try {
	// Oeffne Verbindung auf Port 67 - Port ist generell beliebig
	// Desktop Applikation muss allerdings den gleichen Port benutzen
	rCon = (RadiogramConnection) Connector.open("radiogram://broadcast:" + HOST_PORT);
	dg = rCon.newDatagram(50);  // Testsenden
	} 
catch (Exception e) {
	System.err.println("Exception " + e + " beim Verbindungsaufbau.");
	notifyDestroyed();
}
\end{lstlisting}

Im oben aufgelisteten Codeausschnitt erstellt der SunSPOT eine Radiogram-Connection als Broadcast auf Port 67. Um die Verbindung zu testen, wird ein kleines Datagram Package erstellt und versendet. Falls die Verbindung fehlschlagen sollte, wird eine Fehlermeldung ausgegeben.

\begin{lstlisting}[language=Java,caption={Schleife zur Detektion von Bewegung},label=lst:detektion,frame=single] 
while (true) {
	try {
		if (isMoving()) {
		// Uebermittle die aktuelle Zeit
		long now = System.currentTimeMillis();

		// LED leuchtet, um zu zeigen, dass sich der Sensor bewegt hat
		led.setRGB(255, 255, 255);
		led.setOn();
		Utils.sleep(50);
		led.setOff();

		// Zeit wird ins Package geschrieben, Package wird verschickt
		dg.reset();
		dg.writeLong(now);
		rCon.send(dg);
		}
	} 
	catch (Exception e) {
	System.err.println("Exception " + e + " waehrend des Senden eines Paketes.");
	}
}
\end{lstlisting}

In einer Endlosschleife wird festgestellt, ob sich der Sensor bewegt hat. Dies wird mit Hilfe der Funktion 'isMoving()' realisiert. Falls eine Bewegung stattfindet, wird die aktuelle Systemzeit ermittelt und in ein Datagram-Paket geschrieben. Eine einzelne LED am SPOT leuchtet, um die erfahrene Bewegung anzuzeigen. Das Paket wird daraufhin über die aufgebaute Verbindung versendet. Sollte dieses Senden fehlschlagen, wird ebenfalls eine Fehlermeldung ausgegeben.\newpage

\begin{lstlisting}[language=Java,caption={Methode zur Bewegungserkennung},label=lst:movement,frame=single] 
protected boolean isMoving() throws IOException {
	IAccelerometer3D accel = (IAccelerometer3D) Resources.lookup(IAccelerometer3D.class);
	return accel.getAccelZ() < 0 || accel.getAccelZ() > 0.1;
}
\end{lstlisting}

Die Bewegung wird mittels des eingebauten Beschleunigungssensor ermittelt. Erreicht die erfahrene Beschleunigung einen gewissen Schwellwert, gibt die oben gezeigte Subroutine den wert 'true' zurück und zeigt somit an, dass der Sensor entsprechend beschleunigt wurde.\\

Sobald sowohl Desktop-Applikation als auch SPOT-Applikation gestartet wurden, beginnt das Zusammenspiel der beiden Teilimplementierungen. Die Desktop-Applikation wartet darauf, Pakete, welche die SPOT-Applikation versendet, zu empfangen. So wird bei jeder Bewegung die Desktop-Applikation benachrichtigt. Eine einfache Einbruchserkennung ist hiermit realisiert.